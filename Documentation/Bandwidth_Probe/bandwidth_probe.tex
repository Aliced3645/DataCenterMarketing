\documentclass[a4paper,10pt]{article}
\usepackage[utf8x]{inputenc}

%opening
\title{Note on Bandwidth Estimation}
\author{rui zhou}

\begin{document}

\maketitle

 
\section{Target}
We need to way to determine the available bandwidth between two hosts with in our action network.

\section{Solution}
\subsection{Existing Approaches}

There are some traditional ways to do bandwidth estimation:
\begin{enumerate}
 \item 
Passive Estimation\\
Based on the statistics data from previous traffic. This is not quite useful in out data center networks
 \item 
Active Estimation : Package Pair\\
Send two package  through the link we want to test, waiting for the echo. Estimate the available bandwidth 
by examination on the difference of the gaps between the first bits of the sent packages and the echos. 
\item
IGI Algorithm\\
The IGI algorithm sends a sequence of packet trains with increasing initial packet gap. We monitor
the difference between the average output gap and the input gap for each train and use the first
train for which the two are equal. This point is called turning point. At this point, we use the IGI
formula to compute the competing bandwidth. The available bandwidth is obtained by subtracting
the estimated competing traffic bandwidth from an estimate of the bottleneck link bandwidth. 
\end{enumerate}


\subsection{Drawbacks}
Drawbacks of the aforementioned techniques:\\
They all requires sending multiple packets through a period of time, and its accuracy is largely affected by the variation of the  competing 
traffic. The statistics nature of this algorithm make a real-time estimation very hard to achieve.

\subsection{Proposed Solution}
Bandwidth estimation based on network resource allocation scheduler and an variant factor $\alpha$. 


 

\section{Reference}
Estimating Available Bandwidth
Using Packet Pair Probing, Ningning Hu, Peter Steenkiste
September 9, 2002
CMU-CS-02-166\\

Evaluation and Characterization of Available
Bandwidth Probing Techniques, Ningning Hu, Student Member, IEEE, and Peter Steenkiste, Senior Member, IEEE\\


\end{document}
